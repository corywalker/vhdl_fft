\documentclass[12pt]{article}

% PACKAGES
\usepackage[utf8]{inputenc}
\usepackage{amsmath}
\usepackage{amssymb}
\usepackage{float}
\usepackage{graphicx}
\usepackage{enumerate}
\usepackage{mathtools}

% CUSTOM STYLES
\providecommand{\e}[1]{\ensuremath{\times 10^{#1}}}
\DeclarePairedDelimiter\abs{\lvert}{\rvert}%
\restylefloat{table}

% METADATA
\title{FPGA implementation of Cooley-Tukey}
\date{September 13, 2014}
\author{Garrett Massman and Cory Walker}

\begin{document}

  \maketitle

  \subsection*{Overview}
    In this project we will implement the Cooley-Tukey FFT algorithm on a Xilinx FPGA using VHDL.
  \subsection*{Theory}
    Describe the definitions of the DFT and the Cooley-Tukey FFT.
  \subsection*{In Mathematica}
    Describe Mathematica code, how precomputed tables can help us.
  \subsection*{FPGA considerations}
    Describe how we plan to utilize SPI for communication, what type of logical blocks we will use and how they will work together.
  \subsection*{Block diagram}
    Insert GraphViz here
  \subsection*{Roadmap}
    Describe the steps, in order, that need to be done.
  \subsection*{Possible extensions}
    Describe the extra things we might add if we have time.
  \subsection*{Conclusion}

\end{document}
